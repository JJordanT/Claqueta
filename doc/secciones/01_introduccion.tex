

\chapter{Introducción}

Gran parte del entretenimiento audiovisual es el propio contenido, ya sean series, películas,
documentales, recitales, conciertos, etc. Un grupo de personas simplemente se conforma con visualizar el 
contenido, pero existe otro grupo que no satisface su necesidad de entretenimiento solo visionando estos 
productos. Por ello necesitan algo más, ese algo es compartir su experiencia, su opinión acerca del 
producto consumido, con sus allegados y amigos. Pero como es posible que lo hagan con otros usuarios que 
sientan la misma necesidad de debatir o especular sobre la obra audiovisual que han visto.

A través de reseñas. \textit{''Una reseña cinematográfica es una reacción instantánea de un espectador a 
una película. Es como el chisme en una cena, cuando te vuelves hacia la persona de al lado y le 
preguntas '¿qué pensaste?' Una reseña de cine es una respuesta a una experiencia compartida de la 
película, no es una síntesis de los elementos que la componen''}. Definición de reseña por 
\textit{''Roger Ebert''} un reconocido escritor, y critico de cine de Estados Unidos \cite{REwebsite}. 

Esa necesidad de comentar la obra que acababas de visualizar, comenzó a finales del siglo XIX y a 
principios del siglo XX, ya que se popularizó el cine como forma de entretenimiento. Creando así la 
necesidad de comentar, analizar y discutir estas películas. Gracias a la concepción del cine como un 
arte y no solo como un entretenimiento a principio de los años 20 por los críticos franceses, entre los 
que se encontraba \textit{''Louis Delluc''}, periodista, director y critico francés \cite{LouisD}. 
Surgieron entre los años 20 y 30 críticos en otros países como Reino Unido y Estados Unidos como el 
ganador del premio Pulitzer en 1958 por su novela autobiográfica \textit{''James Agee''} \cite{JamesA}. 

En el titulado \begin{otherlanguage}{english}``\textit{A VERY SHORT HISTORY OF FILM 
CRITICISM}''\end{otherlanguage}\cite{3BrothersArticle}\footnote{\url{https://3brothersfilm.com/blog/2018/7/23/7rnnqs20722u62gz2hhwmrmwar1ut4}}. Encontramos como el autor nos cuenta como ha evolucionado la 
crítica cinematográfica desde la época del famoso crítico \textit{''Roger Ebert''} hasta la era de las 
redes sociales. El autor trata de explicar que la crítica cinematográfica ha seguido siendo importante y 
evoluciona a lo largo de los años. Nos remarca que la crítica cinematográfica era más elitista y se 
enfocaba en películas de arte y ensayo, un tipo de cine que no ofrece una narración dramática ni una 
representación histórica del mundo, sino una reflexión sobre el propio medio. Diferenciándose de la 
narrativa clásica en varios puntos, siendo narrativa de arte y ensayo frente a narrativa clásica. 
Casualidad frente a  causalidad, un final abierto frente a un final cerrado, tiempo no lineal frente a 
tiempo literal, conflicto interno frente a conflicto externo, un protagonista pasivo frente a un 
protagonista activo, una realidad ambigua frente a una realidad coherente y el centro de la historia es 
el personaje y no la acción mientras que por la narrativa clásica se centra en la acción y no en el 
personaje. Con el tiempo la crítica se extendió y se añadieron a su abanico las películas comerciales. 
La influencia de \textit{''Roger Ebert''} ayudo a popularizar la crítica cinematográfica en Estados 
Unidos gracias a sus escritos y a sus programas de televisión. 

A lo largo del tiempo, estos críticos han realizado un gran trabajo en la industria cinematográfica, ya 
que no solo han servido de guía para los espectadores, llevándolos hacia las películas más destacadas, 
gracias a su perspectiva crítica sobre estas. Si no que también has desempeñado un papel importante en 
el desarrollo de la teoría del cine y la crítica literaria. Influyendo así en la forma en que se piensa 
y se habla sobre el cine en la cultura general. Se podría decir que el desarrollo del cine fue el que 
permitió que apareciera el mundo de las críticas cinematográficas, siendo una parte importante de la 
industria e importante para construir nuestra parte crítica dirigida a estas obras cinematográficas. Las 
herramientas que se usan para realizar estas reseñas también se han desarrollado a lo largo del tiempo, 
desde las herramientas más básicas de escritura, papel y pluma, pasando por la máquina de escribir, uso 
de grabadoras y cámaras, la llegada del ordenador de sobremesa y llegando hasta el ordenador portátil. 
Estas herramientas han permitido que las opiniones de los críticos lleguen cada vez a más gente en el 
menos tiempo posible. El mayor salto fue la aparición de internet, de manera que podían compartir su 
opinión llegando a casi todas las partes del mundo, incluso gracias a nuevas tecnologías que han ido 
apareciendo como el \begin{otherlanguage}{english}``\textit{streamming}''\end{otherlanguage} o las redes 
sociales comentadas en el artículo mencionado con anterioridad, que ofrecen realizar criticas en tiempo 
real, siendo una herramienta con mucho potencial y al alcance de la mano de cualquier usuario.

De este modo existen varios espacios famosos como IMDb\cite{IMDbWeb}, web por excelencia conocida por 
ser una de las pioneras en internet. Contiene una gran cantidad de información sobre todo lo relacionado 
con el cine y las series. Una masiva información de calidad al alcance de cualquier usuario, desde si 
buscas a un actor en particular, cierto director o simplemente quieres comentar lo mucho que te ha 
entusiasmado una película que acabas de ver. También posee un sistema de ranking mediante el cual los 
usuarios pueden puntuar cada reseña, estableciendo así cierta reputación a usuarios que tengan una 
puntuación decente gracias a sus opiniones fundadas sobre la obra cinematográfica. Otros espacios 
similares serían Rotten Tomatoes\cite{RottenTWeb} o una web española como SensaCine\cite{SSweb} en las 
que podemos realizar todas las actividades mencionadas antes.

Como se mencionó antes, la llegada de internet y las redes sociales ha permitido que la crítica 
cinematográfica sea más accesible y democrática. Pero también ha traído una ingente cantidad de 
información, provocando una sobrecarga de información y una perdida de profundidad crítica, debido a la 
fugacidad y la celeridad de las redes sociales, han conducido a la crítica cinematográfica a que sea 
más superficial y menos reflexiva. Aun así, la crítica cinematográfica sigue siendo importante en la 
época de las redes sociales, ya que puede ayudar a sus consumidores a comprender mejor el mundo en el 
que viven, descubrir películas y directores interesantes. Por lo que el artículo mencionado 
anteriormente concluye afirmando que la critica, a pesar de haber cambiado a lo largo de los años, 
continúa siendo una forma valiosa de arte y análisis cultural, y anima a sus lectores a buscar y 
compartir reseñas sobre obras cinematográficas que visualicen.

\section{Motivación}

Desde muy pequeño siempre me ha gustado ver las cintas VHS que había en mi casa, desde que me las ponía 
mi familia, hasta que yo pude poner una y otra vez esas cintas, llegando a repetir la misma película 
infinidad de veces sin cansancio alguno. De alguna manera el cine es un mundo aparte en el que no hay 
nada más que la película que estás viendo en ese momento, no hay polémicas, problemas, agobios, estrés. 
Te desinhibes al completo y disfrutas de la película con los que te rodean, riendo, llorando, cantando 
o sintiendo, por esos personajes que hacemos nuestros cuando vemos alguna obra cinematográfica. 

Y de alguna manera el poder expresar lo que te ha parecido la obra que has visto y compartirlo con 
aquellas personas que se crucen con tu reseña y tengan la posibilidad de interactuar contigo y discutir 
sobre lo que ambos habéis visto. Me hace volver a esos tiempos en los que estás con amigos o familia 
comentando tranquilamente en casa, una película en la televisión. Al final solo estás viendo ficción, 
comedia, tristeza, alegría, terror, vidas, muertes. Siendo el objetivo de la obra, transmitir ese 
sentimiento y hacer que te evadas de lo demás. También es una herramienta muy fuerte en países con 
algún régimen opresor, siendo una manera de expresión, liberación, comunicación y protesta frente al 
gobierno que los oprime. Otro motivo es la integración cultural, existen idiomas o culturas que se 
pierden a lo largo del tiempo, si es posible resguardar dicho lenguaje y costumbres de aquellas 
culturas, aunque sea ficción, merecerá la pena para que persista en el tiempo y pueda ser visto por 
otros que no vivieron en esos tiempos. Recordando así tanto lo malo como lo bueno de la historia, 
pudiendo aprender de ello.

Y las reseñas sobre esas obras tendrían el poder de transmitir al resto del mundo el sentimiento 
producido por ese cine y conseguir que el público contribuya a la causa, presionando a la parte 
opresora o simplemente divulgando para que más gente disfrute de esas obras audiovisuales. Por tanto, 
el poder de la escritura, de la reseña en este caso, no es una minucia.


Cita de \textit{''Neil Gaiman''}\cite{NeilG} famoso escritor británico de ficción, novelas gráficas,
películas, teatro de audio...
\begin{verbatim}
    ``Solo escribiendo puedo hacer 
    que el mundo sea lo que debería ser''.
\end{verbatim}

De esta manera, toda persona que disfrute con el cine tendrá la oportunidad de compartir su experiencia 
tras la visualización de cualquier contenido, e instigar a otras personas a ver la obra o incluso a 
comentar lo que le ha parecido. Por ello, gracias a la época tecnológica en la que vivimos, es posible 
encontrar una solución a esa necesidad de cualquier usuario de manifestar su postura frente a las obras 
audiovisuales que consuma. Por lo que el motivo de este TFG es proporcionar una solución informática 
frente a esta necesidad de los usuarios, que sea útil, de manera que anime al usuario a usarla en su 
día a día para satisfacer su necesidad de compartir su experiencia y comentarlo con el resto de 
usuarios, sencilla y accesible para que todo usuario pueda utilizarla y acceder a ella de la forma más 
simple posible. Más debe ser una solución rentable y sostenible a largo plazo, y dispuesta a cambios 
suscitados por los usuarios que consumen dicha solución, para hacerla cada vez más robusta y apegada a 
las necesidades constantes de los usuarios. Pudiendo así ampliar la funcionalidad de la solución una 
vez resueltas las necesidades básicas.

Este trabajo está dividido en tres partes diferenciadas:
\begin{itemize}
    \item En los \textbf{capítulos del 1 al 3} se debe comprender el trabajo y el contexto, describiendo 
    posteriormente el problema que se quiere resolver y los objetivos que se desean lograr. Tras esto, 
    en el estado del arte se analizarán las alternativas existentes en el mercado actualmente en 
    relación con espacios que permitan realizar críticas cinematográficas y compartirlas con los demás 
    usuarios, entendiendo de una mejor manera el dominio del problema.
    \item La segunda parte, construido por los \textbf{capítulos 4 y 5}, se concentran los matices 
    necesarios para llevar a cabo el proyecto, planificando el desarrollo, explicando la metodología a 
    usar, las distintas herramientas que se usaran en el proyecto y preparar el terreno para llevar a 
    cabo la implementación de la solución estudiada.
    \item Para terminar, en los \textbf{capítulos 6 y 7} se realizará todo el trabajo de desarrollo y se 
    evaluará la resolución de la propuesta, valorando si ha cumplido con las competencias y objetivos 
    propuestos.
\end{itemize}

Este proyecto es software libre, y está liberado con la licencia \cite{gplv3}. Y se puede encontrar en 
GitHub en este repositorio público \footnote{\url{https://github.com/JoseJordanF/Claqueta}}