\chapter{Descripción del problema}

En este capítulo se va a profundizar más en el problema que se plateaba en el primer capítulo y se expondrán los objetivos que se pretenden lograr en este proyecto

\section{Problema a resolver}
Actualmente, vivimos en una sociedad apegada a las redes sociales, las cuales han permitido la comunicación entre usuarios que en su vida jamás habrían podido encontrarse. En sí, el intercambio de ideas y el diálogo han sido expandidos a niveles mundiales, prácticamente no existe ninguna barrera que no te permita comunicarte con quien quieras. 

Precisamente, como mencionábamos en el primer capítulo, desde que nación la crítica cinematográfica más y más usuarios se han unido tanto al arte del cine como a divulgar sus opiniones sobre las obras que ven. Antes de la llegada de las redes sociales, compartir tu experiencia a menudo era complicado. Más allá de tus allegados consumidores de cine, no quedaba mucha gente más a la que le interesara tu crítica. Incluso si no compartes el mismo gusto por el cine de tu familia o amigos y prefieres otro género cinematográfico, ni siquiera podías compartir tu opinión. Ya que no era una película que le interesara a los demás. 

Con la llegada de las redes sociales paso hacer algo más fácil, todo el mundo puede decir lo que quiera en sus redes y por algoritmos de recomendaciones de estas redes, seguramente usuarios afines a esos géneros cinematográficos o a esas películas, encontraran tu contenido. Pudiendo conversar libremente con esos usuarios. El problema radica en que es una red social para socializar cualquier cosa, por eso se necesita un espacio dirigido a los usuarios amantes de contenidos audiovisuales como el cine y las series. Así podrías acotar dicho espacio y dedicarlo únicamente a esta infinidad de usuarios, permitiendo que interactúen sin la necesidad de ver contenido que no desean.

Resolviendo el problema, la necesidad de compartir con los demás, tu reseña sobre algún contenido que acabas de visualizar. Una vez en este espacio el mayor problema se ha podido solucionar, pero esta necesidad de que los usuarios expresen sus opiniones no es más que el principio, ya que esta necesidad viene acompañada de aspectos más personalizables, ya que se encuentran en un espacio destinado exclusivamente a ese propósito concreto de realizar una crítica del contenido que elijan. Lo cual incide en más problemas de personalización de dicho espacio, como la facilidad de adaptar el espacio para disfrutar del contenido a tu gusto y obtener recomendaciones por parte del espacio relacionadas con el contenido que consumes. Y que finalmente este espacio logre satisfacer toda necesidad del usuario, disfrutando así del espacio y llegando a promocionar su uso a usuarios cercanos a él.

Por esta razón es muy importante escuchar al cliente, ya que son ellos los que van a explotar el producto, esto se verá más profundamente en el capítulo sobre la planificación donde se hablara de las ``\textit{historias de usuario}''.

A continuación se describirán los objetivos que se desean alcanzar para solucionar el problema que se acaba de plantear en este capítulo.

\section{Objetivos}

Tras haber comprendido mejor el problema y lo que supone, se expondrán los objetivos que se pretenden lograr a través de la solución a dicho problema.

\begin{itemize}
    \item \textbf{OBJ1} Llegar a diseñar y desarrollar un espacio software en el que sea posible compartir tu opinión sobre cualquier contenido cinematográfico, ya sean películas o series. Con un sistema de reseñas fluido y en el que puedas manifestar tu experiencia y apreciar la de otros, pudiendo señalar tus favoritas.
    \item \textbf{OBJ2} Que dicho espacio sea atractivo y cómodo para el usuario a través de su interfaz y manejo, llamando a su uso para cubrir la necesidad del cliente.
    \item \textbf{OBJ3} Diseñar y desarrollar un sistema de recomendaciones para los usuarios, según el contenido que consumen, de manera que puedan manifestar las obras cinematográficas que les gusten. Siendo estas una referencia para sugerirle contenido similar y animar a consumir más cine.
    \item \textbf{OBJ4} Diseñar y desarrollar un perfil por el que el usuario se sienta identificado y presente a quien lo visita, sus gustos y sus reseñas más valoradas o más relevantes.
    \item \textbf{OBJ5} Tener la posibilidad de añadir cualquier sugerencia apegada al mundo del cine si la comunidad así lo requiere, viendo si es mínimamente viable tras su diseño para terminar siendo desarrollado.
    \item \textbf{OBJ6} La herramienta debe ser diseñada de manera que el producto final sea veloz y eficaz, para agradar lo máximo posible al usuario, intentando consumir siempre menos recursos y ayudando a la posibilidad del \textbf{OBJ5}.
\end{itemize}

Estos serán los objetivos que el proyecto pretende alcanzar, de esta manera en el último capítulo se podrá analizar si realmente la solución dada ha podido lograr dichos objetivos.